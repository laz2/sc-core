% -----------------------------------------------------------------------------
% This source file is part of OSTIS (Open Semantic Technology for Intelligent Systems)
% For the latest info, see http://www.ostis.net
% 
% Copyright (c) 2012 OSTIS
% 
%
% OSTIS is free software: you can redistribute it and/or modify
% it under the terms of the GNU Lesser General Public License as published by
% the Free Software Foundation, either version 3 of the License, or
% (at your option) any later version.
% 
% OSTIS is distributed in the hope that it will be useful,
% but WITHOUT ANY WARRANTY; without even the implied warranty of
% MERCHANTABILITY or FITNESS FOR A PARTICULAR PURPOSE.  See the
% GNU Lesser General Public License for more details.
% 
% You should have received a copy of the GNU Lesser General Public License
% along with OSTIS.  If not, see <http://www.gnu.org/licenses/>.
% -----------------------------------------------------------------------------

\section{Формализация в sc-коде}

\subsection{От теории множеств к sc-коду}

\begin{frame}{Множество}
  \begin{center}
    \[ S = \{ \}; \]

    \objeqv  

    \begin{figure}
      \includegraphics{basic/Set}
    \end{figure}
  \end{center}

  \begin{itemize}
  \item такой sc-элемент называется \textbf{константным sc-узлом}
  \item подпись рядом с sс-элементом называется его идентификатором
  \item при помощи такого sc-элемента может обозначаться внешний
    объект, а не только множество
  \end{itemize}
\end{frame}

\begin{frame}{Принадлежность элемента множеству}
  \begin{center}
    \[ a \in S; \] или \[ S = \{ a \}. \]

    \objeqv  

    \begin{figure}
      \includegraphics{basic/El_in_set}
    \end{figure}
  \end{center}

  \begin{itemize}
  \item такой sc-элемент называется \textbf{позитивной константной sc-дугой}
  \end{itemize}
\end{frame}

\begin{frame}{Непринадлежность элемента множеству}
  \begin{center}
    \[ a \notin S \]

    \objeqv  

    \begin{figure}
      \includegraphics{basic/El_not_in_set}
    \end{figure}
  \end{center}

  \begin{itemize}
  \item такой sc-элемент называется \textbf{негативной константной sc-дугой}
  \end{itemize}
\end{frame}

\begin{frame}{Нечеткая принадлежность элемента множеству}
  \begin{figure}
    \centering
    \includegraphics{basic/El_fuzzy_in_set}
  \end{figure}

  \begin{itemize}
  \item такой sc-элемент называется \textbf{нечеткой константной sc-дугой}
  \item используется, когда неизвестна позитивность/негативность sc-дуги
  \end{itemize}
\end{frame}

\begin{frame}{А теперь усложним пример...}
  \begin{center}
    \[ S = \{S1, S2, S2, a\}; \]
    \[ S1 = \{a, b\}; \]
    \[ S2 = \{b, c, d\}. \]
  \end{center}

  \objeqv

  \begin{figure}
    \includegraphics[scale=0.6]{basic/Complex_sets_example}
  \end{figure}
\end{frame}

\begin{frame}{Мощь семантических сетей...}
  \begin{center}
    Честно сказать, я затрудняюсь с ходу написать здесь что-то
    полезное.
  \end{center}

  \objeqv  

  \begin{figure}
    \centering
    \includegraphics{basic/Very_complex_sets_example}
  \end{figure}

  \begin{itemize}
  \item sc-элементы могут не иметь идентификаторов
  \item sc-дуги также могут быть элементами множеств
  \end{itemize}
\end{frame}

\begin{frame}{Кортеж}
  \begin{center}
    \[ T = <a, b, c> \]
    
    \objeqv

    \begin{figure}
      \includegraphics[scale=0.65]{basic/Tuple}
    \end{figure}
  \end{center}

  \begin{itemize}
  \item sc-элемент в виде кружка с крестом внутри называется
    \textbf{атрибутом} и выражает роль объекта в рамках множества
  \item атрибуты \idtf{1\_}, \idtf{2\_}, \idtf{3\_} и т.д. называются
    порядковыми
  \item идентификатор атрибута должен заканчиваться символом `\_`
  \item кортеж в sc-тексте - это множество, каждый элемент которого
    принадлежит ему с указанием определенной роли
  \end{itemize}
\end{frame}

\begin{frame}{Кортеж с нечисловыми атрибутами}
  \begin{center}
    \begin{eqnarray*}
      T = <\cyrmathit{первый}\_: a, \cyrmathit{второй}\_: b, \\
      \cyrmathit{третий}\_: \cyrmathit{дополнительный}\_: c>
    \end{eqnarray*}
    
    \objeqv

    \begin{figure}
      \includegraphics[scale=0.7]{basic/Tuple_with_custom_attrs}
    \end{figure}
  \end{center}

  \begin{itemize}
  \item ролей (атрибутов) у вхождения элемента в множество может быть несколько
  \end{itemize}
\end{frame}

\begin{frame}[shrink=20]{Отношение}
  \begin{center}
    \[ R = \{ <a, b>, <c, d> \} \]
    
    \objeqv

    \begin{figure}
      \includegraphics[scale=0.8]{basic/Relation}
    \end{figure}
  \end{center}

  \begin{itemize}
  \item для обозначения отношений используется кружок с наклонным
    крестом внутри
  \item элемент в виде кружка с горизонтальной чертой внутри
    называется связкой и выражает связь между своими компонентами
  \item в sc-коде отношение - это множество связок
  \item идентификатор отношения в sc-коде должен обязательно
    заканчиваться символом "*"
  \end{itemize}
\end{frame}

\begin{frame}{Бинарная ориентированная пара}
  \begin{center}
    \begin{figure}
      \includegraphics{basic/Bin_ord_tuple}
    \end{figure}

    \objeqv

    \begin{figure}
      \includegraphics{basic/Bin_ord_pair}
    \end{figure}
  \end{center}

  \begin{itemize}
  \item так как бинарные связки используются очень часто, то для них
    введен специальный элемент - бинарная ориентированная пара
  \end{itemize}
\end{frame}

\begin{frame}{Отношение c бинарными ориентированными парами}
  \begin{center}
    \begin{figure}
      \includegraphics[scale=0.8]{basic/Relation}
    \end{figure}
    
    \objeqv

    \begin{figure}
      \includegraphics[scale=0.8]{basic/Relation_with_bin_ord_pairs}
    \end{figure}
  \end{center}
\end{frame}

\begin{frame}{Бинарная неориентированная пара}
  \begin{center}
    \begin{figure}
      \includegraphics{basic/Bin_unord_tuple}
    \end{figure}

    \objeqv

    \begin{figure}
      \includegraphics{basic/Bin_unord_pair}
    \end{figure}
  \end{center}

  \begin{itemize}
  \item связка вполне может быть неориентированной
  \item связка - это не всегда кортеж (ориентированное множество)
  \end{itemize}
\end{frame}

\begin{frame}[shrink=10]{Отношение \idtf{включение*} и \idtf{строгое
      включение*}}
  \begin{center}
    \begin{eqnarray*}
      S_1 = \{ a, b, c, d \}; \\
      S_2 = \{ b, c \}; \\
      S_1 \supset S_2; \\
      S_1 \supseteq S_2.
    \end{eqnarray*}

    \objeqv

    \begin{figure}
      \includegraphics[scale=0.7]{basic/Relation_Inclusion_and_Strict_inclusion}
    \end{figure}
  \end{center}

  \begin{itemize}
  \item sc-элементы с одинаковыми идентификаторами склеиваются
  \end{itemize}
\end{frame}

\begin{frame}[shrink=10]{Отношение \idtf{объединение*}}
  \begin{center}
    \begin{eqnarray*}
      S_1 = \{ S_2, a, b \}; \\
      S_2 = \{ b, c, d \}; \\
      S = S_1 \cap S_2 = \{ a, b, c, d, S_2 \}.
    \end{eqnarray*}

    \objeqv

    \begin{figure}
      \includegraphics[scale=0.7]{basic/Relation_Union}
    \end{figure}
  \end{center}
\end{frame}

\begin{frame}[shrink=10]{Отношение \idtf{пересечение*}}
  \begin{center}
    \begin{eqnarray*}
      S_1 = \{ S_2, a, b \}; \\
      S_2 = \{ b, c, d \}; \\
      S = S_1 \cup S_2 = \{ b \}.
    \end{eqnarray*}

    \objeqv

    \begin{figure}
      \includegraphics[scale=0.7]{basic/Relation_Intersection}
    \end{figure}
  \end{center}
\end{frame}

\begin{frame}[shrink=10]{Отношение \idtf{разбиение*}}
  \begin{center}
    \begin{figure}
      \includegraphics[scale=0.7]{basic/Relation_Partition}
    \end{figure}
  \end{center}
\end{frame}


\begin{frame}{Доп. элемент SCg: Шина}
  \begin{center}
    \begin{figure}
      \includegraphics{basic/Bus_example}
    \end{figure}
  \end{center}
\end{frame}

\begin{frame}{Доп. элемент SCg: Контур}
  \begin{center}
    \begin{figure}
      \includegraphics[scale=0.55]{basic/Contour_example}
    \end{figure}
  \end{center}
\end{frame}

\subsection{Формализация в sc-коде}

\begin{frame}{Виды понятий в sc-коде}
  Понятие (sc-код) - это множество, все элементы которого называются
  экземплярами (объектами) данного понятия.


  При формализации предметной области в sc-коде выделяют следующие
  виды понятий:
  \begin{itemize}
  \item относительное понятие (всегда отношение):
    \begin{itemize}
    \item отношение общего вида
    \item ролевое отношение
    \end{itemize}
  \item абсолютное понятие (все то, что не относительное понятие)
  \end{itemize}
\end{frame}

\begin{frame}{Абсолютное понятие в sc-коде}
  Экземпляр абсолютного понятия никогда не выражает связь между объектами.
  
  \begin{center}
    \begin{figure}
      \includegraphics{basic/Absolute_concept_Number}
    \end{figure}
  \end{center}

  \begin{itemize}
  \item sc-элемент \idtf{число} - это абсолютное понятие, которое
    выражает множество всех чисел
  \item для обозначения абсолютного понятия используется sc-элемент со
    звездой внутри
  \end{itemize}
\end{frame}

\begin{frame}{Абсолютное понятие в sc-коде (sc-структура)}
  Экземпляр абсолютного понятия может состоять из частей, тогда он
  является sc-структурой.
  
  \begin{center}
    \begin{figure}
      \includegraphics{basic/Absolute_concept_Complex_number}
    \end{figure}
  \end{center}

  \begin{itemize}
  \item sc-элемент с точкой внутри - это sc-структура (для простоты
    понимания проведите аналогию со структурами в языках
    программирования)
  \end{itemize}
\end{frame}

\begin{frame}{Относительное понятие в sc-коде (отношение общего вида)}
  \begin{center}
    \begin{figure}
      \includegraphics{basic/Relational_concept_Greater}
    \end{figure}
  \end{center}
\end{frame}

\begin{frame}{Относительное понятие в sc-коде (ролевое отношение)}
  Ролевое (атрибутивное) отношение - это атрибут.

  \begin{center}
    \begin{figure}
      \includegraphics{basic/Absolute_concept_Complex_number}
    \end{figure}
  \end{center}
\end{frame}


%%% Local Variables: 
%%% mode: latex
%%% TeX-master: "sc_code_in_examples"
%%% End: 
