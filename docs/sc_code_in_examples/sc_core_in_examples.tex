% -----------------------------------------------------------------------------
% This source file is part of OSTIS (Open Semantic Technology for Intelligent Systems)
% For the latest info, see http://www.ostis.net
% 
% Copyright (c) 2012 OSTIS
% 

% OSTIS is free software: you can redistribute it and/or modify
% it under the terms of the GNU Lesser General Public License as published by
% the Free Software Foundation, either version 3 of the License, or
% (at your option) any later version.
% 
% OSTIS is distributed in the hope that it will be useful,
% but WITHOUT ANY WARRANTY; without even the implied warranty of
% MERCHANTABILITY or FITNESS FOR A PARTICULAR PURPOSE.  See the
% GNU Lesser General Public License for more details.
% 
% You should have received a copy of the GNU Lesser General Public License
% along with OSTIS.  If not, see <http://www.gnu.org/licenses/>.
% -----------------------------------------------------------------------------

\documentclass[hyperref={pdftex,unicode}]{beamer}

\usepackage[T2A]{fontenc}
\usepackage[english,russian]{babel}
\usepackage[utf8]{inputenc}

% Подключаем математические формулы
\usepackage{amsmath}
\usepackage{amsfonts}
\usepackage{amssymb}

\usepackage{cmap}               % Поиск по русским буквам в pdf

\usepackage{rotating}

\usetheme{Warsaw}

\newcommand{\objeqv}{
  \begin{center}
    \begin{sideways}
      \[ \Longleftrightarrow \]
    \end{sideways}
  \end{center}
}

\begin{document}

\title{SC-код в примерах}  
\author{Лазуркин Д.А.}
\date{Минск, 2012} 

\begin{frame}
  \maketitle{}
\end{frame}


\section{Базовые элементы sc-кода}
\begin{frame}{Множество}
  \begin{center}
    \[ S = \{ \}; \]

    \objeqv  

    \begin{figure}
      \includegraphics{Set}
    \end{figure}
  \end{center}

  \begin{itemize}
  \item такой sc-элемент называется константным sc-узлом
  \item подпись рядом с sс-элементом называется его идентификатором
  \item с использованием такого sc-элемента может обозначаться внешний
    объект
  \end{itemize}
\end{frame}


\begin{frame}{Принадлежность элемента множеству}
  \begin{center}
    \[ a \in S \] или \[ S = \{a\} \]

    \objeqv  

    \begin{figure}
      \includegraphics{El_in_set}
    \end{figure}
  \end{center}

  \begin{itemize}
  \item такой sc-элемент называется \emph{позитивной} константной sc-дугой
  \end{itemize}
\end{frame}

\begin{frame}{Непринадлежность элемента множеству}
  \begin{center}
    \[ a \notin S \]

    \objeqv  

    \begin{figure}
      \includegraphics{El_not_in_set}
    \end{figure}
  \end{center}

  \begin{itemize}
  \item такой sc-элемент называется \emph{негативной} константной sc-дугой
  \end{itemize}
\end{frame}

\begin{frame}{Нечеткая принадлежность элемента множеству}
  \begin{figure}
    \centering
    \includegraphics{El_fuzzy_in_set}
  \end{figure}

  \begin{itemize}
  \item такой sc-элемент называется \emph{нечеткой} константной sc-дугой
  \item используется, когда неизвестна позитивность/негативность sc-дуги
  \end{itemize}
\end{frame}

\begin{frame}{А теперь усложним пример...}
  \begin{center}
    \[ S = \{S1, S2, S2, a\}; \]
    \[ S1 = \{a, b\}; \]
    \[ S2 = \{b, c, d\}. \]
  \end{center}

  \objeqv

  \begin{figure}
    \includegraphics[scale=0.6]{Complex_sets_example}
  \end{figure}
\end{frame}

\begin{frame}{Мощь семантических сетей...}
  \begin{center}
    Честно сказать, я затрудняюсь с ходу написать здесь что-то
    полезное.
  \end{center}

  \objeqv  

  \begin{figure}
    \centering
    \includegraphics{Very_complex_sets_example}
  \end{figure}

  \begin{itemize}
  \item sc-элементы могут не иметь идентификаторов
  \item sc-дуги также могут быть элементами множеств
  \end{itemize}
\end{frame}

\begin{frame}{Кортеж}
  \begin{center}
    \[ T = <a, b, c> \]
    
    \objeqv

    \begin{figure}
      \includegraphics{Set}
    \end{figure}
  \end{center}

  \begin{itemize}
  \item sc-элемент в виде кружка с горизонтальной чертой называется
    \emph{связкой}, так как выражает связь между объектами
  \item sc-элемент в виде кружка с крестом внутри называется
    \emph{атрибутом} и выражает роль объекта в рамках множества
  \item атрибуты \emph{1\_}, \emph{2\_}, \emph{3\_} и т.д. называются порядковыми
  \item идентификатор атрибута должен заканчиваться символом `_`
  \end{itemize}
\end{frame}

\end{document}
