% -----------------------------------------------------------------------------
% This source file is part of OSTIS (Open Semantic Technology for Intelligent Systems)
% For the latest info, see http://www.ostis.net
% 
% Copyright (c) 2012 OSTIS
% 
%
% OSTIS is free software: you can redistribute it and/or modify
% it under the terms of the GNU Lesser General Public License as published by
% the Free Software Foundation, either version 3 of the License, or
% (at your option) any later version.
% 
% OSTIS is distributed in the hope that it will be useful,
% but WITHOUT ANY WARRANTY; without even the implied warranty of
% MERCHANTABILITY or FITNESS FOR A PARTICULAR PURPOSE.  See the
% GNU Lesser General Public License for more details.
% 
% You should have received a copy of the GNU Lesser General Public License
% along with OSTIS.  If not, see <http://www.gnu.org/licenses/>.
% -----------------------------------------------------------------------------

\documentclass[hyperref={pdftex,unicode}]{beamer}

\usepackage{latexsym}
\usepackage[warn]{mathtext}     % русские буквы буквы в формулах (с предупреждением)
\usepackage[T2A]{fontenc}
\usepackage[english,russian]{babel}
\usepackage[utf8]{inputenc}

% Подключаем математические формулы
\usepackage{amsmath}
\usepackage{amsfonts}
\usepackage{amssymb}

\usepackage{cmap}               % Поиск по русским буквам в pdf

\usepackage[unicode=true]{hyperref}

\usepackage{rotating}

\usetheme{Warsaw}

\newcommand{\objeqv}{
  \begin{center}
    \begin{sideways}
      \[ \Longleftrightarrow \]
    \end{sideways}
  \end{center}
}

\begin{document}

\title{SC-код в примерах}  
\author{Лазуркин Д.А.}
\date{Минск, 2012} 

\begin{frame}
  \maketitle{}
\end{frame}


\section{От теории множеств к sc-коду}
\begin{frame}{Множество}
  \begin{center}
    \[ S = \{ \}; \]

    \objeqv  

    \begin{figure}
      \includegraphics{Set}
    \end{figure}
  \end{center}

  \begin{itemize}
  \item такой sc-элемент называется константным sc-узлом
  \item подпись рядом с sс-элементом называется его идентификатором
  \item с использованием такого sc-элемента может обозначаться внешний
    объект
  \end{itemize}
\end{frame}


\begin{frame}{Принадлежность элемента множеству}
  \begin{center}
    \[ a \in S \] или \[ S = \{a\} \]

    \objeqv  

    \begin{figure}
      \includegraphics{El_in_set}
    \end{figure}
  \end{center}

  \begin{itemize}
  \item такой sc-элемент называется \emph{позитивной} константной sc-дугой
  \end{itemize}
\end{frame}

\begin{frame}{Непринадлежность элемента множеству}
  \begin{center}
    \[ a \notin S \]

    \objeqv  

    \begin{figure}
      \includegraphics{El_not_in_set}
    \end{figure}
  \end{center}

  \begin{itemize}
  \item такой sc-элемент называется \emph{негативной} константной sc-дугой
  \end{itemize}
\end{frame}

\begin{frame}{Нечеткая принадлежность элемента множеству}
  \begin{figure}
    \centering
    \includegraphics{El_fuzzy_in_set}
  \end{figure}

  \begin{itemize}
  \item такой sc-элемент называется \emph{нечеткой} константной sc-дугой
  \item используется, когда неизвестна позитивность/негативность sc-дуги
  \end{itemize}
\end{frame}

\begin{frame}{А теперь усложним пример...}
  \begin{center}
    \[ S = \{S1, S2, S2, a\}; \]
    \[ S1 = \{a, b\}; \]
    \[ S2 = \{b, c, d\}. \]
  \end{center}

  \objeqv

  \begin{figure}
    \includegraphics[scale=0.6]{Complex_sets_example}
  \end{figure}
\end{frame}

\begin{frame}{Мощь семантических сетей...}
  \begin{center}
    Честно сказать, я затрудняюсь с ходу написать здесь что-то
    полезное.
  \end{center}

  \objeqv  

  \begin{figure}
    \centering
    \includegraphics{Very_complex_sets_example}
  \end{figure}

  \begin{itemize}
  \item sc-элементы могут не иметь идентификаторов
  \item sc-дуги также могут быть элементами множеств
  \end{itemize}
\end{frame}

\begin{frame}[shrink=20]{Кортеж}
  \begin{center}
    \[ T = <a, b, c> \]
    
    \objeqv

    \begin{figure}
      \includegraphics{Tuple}
    \end{figure}
  \end{center}

  \begin{itemize}
  \item sc-элемент в виде кружка с крестом внутри называется
    \emph{атрибутом} и выражает роль объекта в рамках множества
  \item атрибуты \emph{1\_}, \emph{2\_}, \emph{3\_} и т.д. называются
    порядковыми
  \item идентификатор атрибута должен заканчиваться символом `\_`
  \item кортеж в sc-тексте - это множество, каждый элемент которого
    принадлежит ему с указанием определенной роли
  \end{itemize}
\end{frame}

\begin{frame}{Кортеж с нечисловыми атрибутами}
  \begin{center}
    \[ T = <\cyrmathit{первый}\_: a, \cyrmathit{второй}\_: b, \cyrmathit{третий}\_: c> \]
    
    \objeqv

    \begin{figure}
      \includegraphics{Tuple_with_custom_attrs}
    \end{figure}
  \end{center}
\end{frame}

\begin{frame}[shrink=20]{Отношение}
  \begin{center}
    \[ R = \{ <a, b>, <c, d> \} \]
    
    \objeqv

    \begin{figure}
      \includegraphics[scale=0.8]{Relation}
    \end{figure}
  \end{center}

  \begin{itemize}
  \item для обозначения отношений используется кружок с наклонным
    крестом внутри
  \item элемент в виде кружка с горизонтальной чертой внутри
    называется связкой и выражает связь между своими компонентами
  \item в sc-коде отношение - это множество связок
  \end{itemize}
\end{frame}

\section{Углубляемся в sc-код}
\begin{frame}{Бинарная ориентированная пара}
  \begin{center}
    \begin{figure}
      \includegraphics{Bin_ord_tuple}
    \end{figure}

    \objeqv

    \begin{figure}
      \includegraphics{Bin_ord_pair}
    \end{figure}
  \end{center}

  \begin{itemize}
  \item так как бинарный связки используются очень часто, то для них
    введенный специальный элемент - бинарная ориентированная пара
  \end{itemize}
\end{frame}

\begin{frame}{Бинарная неориентированная пара}
  \begin{center}
    \begin{figure}
      \includegraphics{Bin_unord_tuple}
    \end{figure}

    \objeqv

    \begin{figure}
      \includegraphics{Bin_unord_pair}
    \end{figure}
  \end{center}

  \begin{itemize}
  \item связка вполне может быть неориентированной
  \end{itemize}
\end{frame}

\section{Теория графов в sc-коде}
\begin{frame}{Поиск одного из минимальных путей в неориентированном графе}
  Теперь мы займемся формализацией c использованием sc-кода данных для
  волнового алгоритма поиска одного из минимальных путей в
  неориентированном графе.

  Вся дальнейшая формализация основыввает на базе знаний по теории
  графов
  \href{http://ostisgraphstheo.sourceforge.net/index.php/Заглавная_страница}{OSTIS Graphs Theory}.
\end{frame}

\subsection{Граф в sc-коде}
\begin{frame}{Представление неориентированного графа}
  Начнем мы с представления в sc-коде неориентированного графа $G$:

  \begin{figure}
    \centering
    \includegraphics[scale=0.7]{graph_theory/Undirected_graph_no_scg}
  \end{figure}
\end{frame}

\begin{frame}{Классический математический способ задания графа}
  Классический математический способ задания графа $G$ будет выглядеть
  следующим образом:
  
  \[ G = <Vertex, Edge>; \]
  \[ Vertex = \{ A, B, C, E, D, F, K \}; \]
  \[ Edge = \{ \{A, B\}, \{A, C\}, \{C, E\}, \{C, D\}, \{B, E\}, \{E, F\} \}. \]
\end{frame}

\begin{frame}{Абсолютное понятия `Неориентированный граф`}
  Для представления неориентированных графов в sc-коде введем
  абсолютное понятия \emph{неориентированный граф}, т.е. множество
  всех неориентированных графов.
  Тогда граф G в sc-коде будет выглядеть следующим образом...
\end{frame}

\begin{frame}{Классический способ задания неориентированного графа (sc-код)}
  \begin{figure}
    \centering
    \includegraphics[scale=0.6]{graph_theory/Undirected_graph_classic}
  \end{figure}
\end{frame}

\begin{frame}{Основной способ задания неориентированного граф}
  Давайте введем два относительных понятия (ролевых отношения)
  \emph{вершина\_} и \emph{ребро\_}, тогда граф $G$ на языке теории множеств можно
  задать следующим образом:

  \begin{eqnarray*}
    G = <\cyrmathit{вершина}\_: A, \cyrmathit{вершина}\_: B, \\
    \cyrmathit{вершина}\_: C, \cyrmathit{вершина}\_: E, \\
    \cyrmathit{вершина}\_: D, \cyrmathit{вершина}\_: F, \\
    \cyrmathit{вершина}\_: K, \cyrmathit{ребро}\_: \{A, B\}, \\
    \cyrmathit{ребро}\_: \{A, C\}, \cyrmathit{ребро}\_: \{C, E\}, \\
    \cyrmathit{ребро}\_: \{C, D\}, \cyrmathit{ребро}\_: \{B, E\}, \\
    \cyrmathit{ребро}\_: \{E, F\}>.
  \end{eqnarray*}
\end{frame}

\begin{frame}{Основная способ задания неориентированного графа (sc-код)}
  \begin{figure}
    \centering
    \includegraphics[scale=0.6]{graph_theory/Undirected_graph_main}
  \end{figure}
\end{frame}

\begin{frame}{Представление ориентированного графа}
  Теперь попробуем представить в sc-коде ориентированного графа $G_d$:

  \begin{figure}
    \centering
    \includegraphics[scale=0.7]{graph_theory/Directed_graph_no_scg}
  \end{figure}
\end{frame}

\begin{frame}{Представление неориентированного графа (доп. понятия)}
  Для представления графа $G_d$ введем абсолютное понятие
  \emph{ориентированный граф} и относительное понятия (ролевое отношение)
  \emph{дуга\_}.
\end{frame}

\begin{frame}{Представление ориентированного графа (sc-код)}
  \begin{figure}
    \centering
    \includegraphics[scale=0.6]{graph_theory/Directed_graph}
  \end{figure}
\end{frame}

\begin{frame}{Сокращенная форма представления графа (sc-код)}
  Для удобства восприятия графа человеком (\emph{но не машиной})
  будем в дальнейшем максимальным образом использовать сокращенную
  форму задания графа:
  
  \begin{figure}
    \centering
    \includegraphics[scale=0.6]{graph_theory/Undirected_graph_short_form}
  \end{figure}
\end{frame}

\subsection{Маршрут в sc-коде}


\end{document}
