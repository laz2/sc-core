% -----------------------------------------------------------------------------
% This source file is part of OSTIS (Open Semantic Technology for Intelligent Systems)
% For the latest info, see http://www.ostis.net
% 
% Copyright (c) 2012 OSTIS
% 
%
% OSTIS is free software: you can redistribute it and/or modify
% it under the terms of the GNU Lesser General Public License as published by
% the Free Software Foundation, either version 3 of the License, or
% (at your option) any later version.
% 
% OSTIS is distributed in the hope that it will be useful,
% but WITHOUT ANY WARRANTY; without even the implied warranty of
% MERCHANTABILITY or FITNESS FOR A PARTICULAR PURPOSE.  See the
% GNU Lesser General Public License for more details.
% 
% You should have received a copy of the GNU Lesser General Public License
% along with OSTIS.  If not, see <http://www.gnu.org/licenses/>.
% -----------------------------------------------------------------------------

\section{Теория графов в sc-коде}

\begin{frame}{Поиск одного из минимальных путей в неориентированном графе}
  Теперь мы займемся формализацией c использованием sc-кода данных для
  волнового алгоритма поиска одного из минимальных путей в
  неориентированном графе.

  Вся дальнейшая формализация основыввает на базе знаний по теории
  графов
  \href{http://ostisgraphstheo.sourceforge.net/index.php/Заглавная_страница}{OSTIS Graphs Theory}.
\end{frame}

\subsection{Граф в sc-коде}

\begin{frame}{Представление неориентированного графа}
  Начнем мы с представления в sc-коде неориентированного графа $G$:

  \begin{figure}
    \centering
    \includegraphics[scale=0.7]{graph_theory/Undirected_graph_no_scg}
  \end{figure}
\end{frame}

\begin{frame}{Классический математический способ задания графа}
  Классический математический способ задания графа $G$ будет выглядеть
  следующим образом:
  
  \[ G = <Vertex, Edge>; \]
  \[ Vertex = \{ A, B, C, E, D, F, K \}; \]
  \[ Edge = \{ \{A, B\}, \{A, C\}, \{C, E\}, \{C, D\}, \{B, E\}, \{E, F\} \}. \]
\end{frame}

\begin{frame}{Абсолютное понятия `Неориентированный граф`}
  Для представления неориентированных графов в sc-коде введем
  абсолютное понятия \idtf{неориентированный граф}, т.е. множество
  всех неориентированных графов.
  Тогда граф G в sc-коде будет выглядеть следующим образом...
\end{frame}

\begin{frame}{Классический способ задания неориентированного графа (sc-код)}
  \begin{figure}
    \centering
    \includegraphics[scale=0.6]{graph_theory/Undirected_graph_classic}
  \end{figure}
\end{frame}

\begin{frame}{Основной способ задания неориентированного граф}
  Давайте введем два относительных понятия (ролевых отношения)
  \idtf{вершина\_} и \idtf{ребро\_}, тогда граф $G$ на языке теории множеств можно
  задать следующим образом:

  \begin{eqnarray*}
    G = <\cyrmathit{вершина}\_: A, \cyrmathit{вершина}\_: B, \\
    \cyrmathit{вершина}\_: C, \cyrmathit{вершина}\_: E, \\
    \cyrmathit{вершина}\_: D, \cyrmathit{вершина}\_: F, \\
    \cyrmathit{вершина}\_: K, \cyrmathit{ребро}\_: \{A, B\}, \\
    \cyrmathit{ребро}\_: \{A, C\}, \cyrmathit{ребро}\_: \{C, E\}, \\
    \cyrmathit{ребро}\_: \{C, D\}, \cyrmathit{ребро}\_: \{B, E\}, \\
    \cyrmathit{ребро}\_: \{E, F\}>.
  \end{eqnarray*}
\end{frame}

\begin{frame}{Основная способ задания неориентированного графа (sc-код)}
  \begin{figure}
    \centering
    \includegraphics[scale=0.6]{graph_theory/Undirected_graph_main}
  \end{figure}
\end{frame}

\begin{frame}{Представление ориентированного графа}
  Теперь попробуем представить в sc-коде ориентированного графа $G_d$:

  \begin{figure}
    \centering
    \includegraphics[scale=0.7]{graph_theory/Directed_graph_no_scg}
  \end{figure}
\end{frame}

\begin{frame}{Представление неориентированного графа (доп. понятия)}
  Для представления графа $G_d$ введем абсолютное понятие
  \idtf{ориентированный граф} и относительное понятия (ролевое отношение)
  \idtf{дуга\_}.
\end{frame}

\begin{frame}{Представление ориентированного графа (sc-код)}
  \begin{figure}
    \centering
    \includegraphics[scale=0.6]{graph_theory/Directed_graph}
  \end{figure}
\end{frame}

\begin{frame}{Сокращенная форма представления графа (sc-код)}
  Для удобства восприятия графа человеком (\emph{но не машиной})
  будем в дальнейшем максимальным образом использовать сокращенную
  форму задания графа:
  
  \begin{figure}
    \centering
    \includegraphics[scale=0.6]{graph_theory/Undirected_graph_short_form}
  \end{figure}
\end{frame}

\subsection{Минимальный путь в sc-коде}

\begin{frame}{Определение маршрута}
  Из книги Ф. Харрари `Теория графов` (глава `Маршруты и связность`, с. 26):
  \begin{quote}
    \textbf{Маршрутом} в графе $G$ называется чередующаяся последовательность
    вершин и ребер $v_0$, $x_1$, $v_1$, …, $v_{n-1}$, $x_n$, $v_n$; эта последовательность
    начинается и кончается вершиной, и каждое ребро последовательности
    инцидентно двум вершинам, одна из которых непосредственно
    предшествует ему, а другая непосредственно следует за
    ним. Указанный маршрут соединяет вершины $v_0$ и $v_n$, и его можно
    обозначить $v_0$, $v_1$, …, $v_n$ (наличие ребер подразумевается).
  \end{quote}
\end{frame}

\begin{frame}{Определение пути}
  Из книги Ф. Харрари `Теория графов` (глава `Маршруты и связность`, с. 26):
  \begin{quote}
    Маршрут называется \textbf{цепью}, если все его ребра различны, и
    \textbf{простой цепью} (\textbf{путем}), если все вершины (а,
    следовательно, и ребра) различны.
  \end{quote}
\end{frame}

\begin{frame}{Пример пути в графе}
  В графе $G$ выделен синем цветом пути $R$, который задается
  последовательностью $A$, $e_2$, $B$, $e_5$, $E$, $e_6$, $F$:
  \begin{figure}
    \centering
    \includegraphics[scale=0.7]{graph_theory/Path_example_no_scg}
  \end{figure}
\end{frame}

\begin{frame}{Путь - это маршрут}
  Так как путь - это маршрут, то, определив более общее понятие, мы
  разберемся с более частным.

  Маршрут - это относительное понятие, которое связывает граф с
  некоторой последовательностью (структурой маршрута).

  Введем бинарное ориентированное отношение \idtf{маршрут*}:
  \begin{figure}
    \centering
    \includegraphics{graph_theory/Example_of_relations_Route_tuple}
  \end{figure}
\end{frame}

\begin{frame}{Различные способы представления маршрута}
  Мы будем рассматривать следующие способы представления структуры
  маршрута:
  \begin{itemize}
  \item подграф
  \item последовательность
  \item соответствие
  \end{itemize}
\end{frame}

\subsubsection{Маршрут, как подграф}

\begin{frame}{Отношение \idtf{подграф*}}
  Пример бинарного ориентированного отношения \idtf{подграф*},
  которое связывает граф $G_1$ с его подграфом $G_2$:
  \begin{figure}
    \centering
    \includegraphics{graph_theory/Relation_Subgraph_example}
  \end{figure}
\end{frame}

\begin{frame}{Маршрут $R$, как подграф}
  \begin{figure}
    \centering
    \includegraphics[scale=0.65]{graph_theory/Route_as_subgraph}
  \end{figure}
\end{frame}

\begin{frame}{Проблема представления маршрута в виде подграфа}
  \begin{figure}
    \centering
    \includegraphics[scale=0.65]{graph_theory/Route_as_subgraph_problem}
  \end{figure}
\end{frame}

\subsubsection{Маршрут, как последовательность}

\begin{frame}{Маршрут $R$, как поcледовательность}
  \begin{figure}
    \centering
    \includegraphics[scale=0.62]{graph_theory/Route_as_sequence}
  \end{figure}
\end{frame}

\subsubsection{Маршрут, как соответствие}

\begin{frame}{От вершин и ребер к посещениям вершин и ребер}
  Снова заглянем в Ф. Харрари `Теория графов` (глава `Маршруты и связность`, с. 26):
  \begin{quote}
    \textbf{Маршрутом} в графе $G$ называется чередующаяся последовательность
    вершин и ребер $v_0$, $x_1$, $v_1$, \dots, $v_{n-1}$, $x_n$, $v_n$; \dots
  \end{quote}

  Давайте сделаем явным в этом определении то, что якобы считается
  само собой разумеющимся:
  \begin{quote}
    \textbf{Маршрутом} в графе $G$ называется чередующаяся последовательность \textbf{посещений}
    вершин и ребер $v_0$, $x_1$, $v_1$, \dots, $v_{n-1}$, $x_n$, $v_n$; \dots
  \end{quote}
\end{frame}

\begin{frame}{Структура маршрут $R$}
  \begin{figure}
    \centering
    \includegraphics[scale=0.58]{graph_theory/Route_as_correspondence_incomplete}
  \end{figure}
\end{frame}

\begin{frame}{Пример отношения \idtf{соответствие*}}
  \begin{figure}
    \centering
    \includegraphics[scale=0.65]{graph_theory/Relation_Correspondece_example}
  \end{figure}
\end{frame}

\begin{frame}{Маршрут $R$, как соответствие}
  \begin{figure}
    \centering
    \includegraphics[scale=0.55]{graph_theory/Route_as_correspondence_final}
  \end{figure}
\end{frame}

\subsubsection{Минимальный путь в графе}

\begin{frame}{Отношения \idtf{цепь*} и \idtf{путь*} (\idtf{простая цепь*})}
  \begin{figure}
    \centering
    \includegraphics{graph_theory/Relations_Trail_Path}
  \end{figure}
\end{frame}

\begin{frame}{Один из минимальных путей в графе $G$}
  \begin{figure}
    \centering
    \includegraphics[scale=0.54]{graph_theory/Path_final}
  \end{figure}
\end{frame}

\subsection{Обобщение различных видов графов}

\begin{frame}{Пример абсолютного понятия \idtf{графовая структура}}
  \begin{figure}
    \centering
    \includegraphics{graph_theory/Graph_structure_example}
  \end{figure}
\end{frame}

\begin{frame}{Иерархия типов \idtf{графовых структур}}
  \begin{figure}
    \centering
    \includegraphics[scale=0.55]{graph_theory/Hierarchy_of_graphs_types}
  \end{figure}
\end{frame}

\begin{frame}{Иерархия ролей элементов \idtf{графовых структур}}
  \begin{figure}
    \centering
    \includegraphics[scale=0.43]{graph_theory/Hierarchy_of_elements_roles_in_graph_structure}
  \end{figure}
\end{frame}

%%% Local Variables: 
%%% mode: latex
%%% TeX-master: "sc_core_in_examples"
%%% End: 
